
\documentclass[12pt,a4paper]{article} % Use A4 paper with a 12pt font size - different paper sizes will require manual recalculation of page margins and border positions

% Generated with LaTeXDraw 2.0.8
% Mon Jun 17 19:00:40 EDT 2013
\usepackage[usenames,dvipsnames]{pstricks}
\usepackage{epsfig}
\usepackage{pst-grad} % For gradients
\usepackage{pst-plot} % For axes
%\usepackage{marginnote} % Required for margin notes
%\usepackage{wallpaper} % Required to set each page to have a background
%\usepackage{lastpage} % Required to print the total number of pages
\usepackage[left=1.3cm,right=4.6cm,top=1.8cm,bottom=4.0cm,marginparwidth=3.4cm]{geometry} % Adjust page margins
\usepackage{amsmath} % Required for equation customization
\usepackage{amssymb} % Required to include mathematical symbols
\usepackage{xcolor} % Required to specify colors by name
\usepackage{amsthm}
\usepackage{float}


\setlength{\parindent}{0cm} % Remove paragraph indentation
\newcommand{\tab}{\hspace*{2em}} % Defines a new command for some horizontal space


\title{Calculus Workshop - Preliminaries}
%----------------------------------------------------------------------------------------

\newtheorem{defn}{Definition}
\newtheorem{example}{Example}
\newtheorem{prop}{Proposition}
\newtheorem{exer}{Exercises}
\newtheorem{thm}{Therorem}
\begin{document}
\maketitle
\section{Preliminaries}
\subsection{Absolute Value}
In the following topics, it is helpful to think of such things as $|x -a|$ as 'distances'.  So, when you see the absolute value of a difference, you really should be thinking of the distance between the numbers on the real line.  For example, consider:
$$|(-2) - 1|$$
We can think of this expression as the distance between $-2$ and $1$ on the number line:
\includegraphics[]{dist.jpg}

Some useful properties of absolute value:
\begin{enumerate}
\item If $|x|\leq a$, then $-a\leq x\leq a$
\item Triangle inequality:  $|x +y| \leq |x| +|y|$
\item Reverse triangle inequality: $||x|-|y|| \leq |x-y|$
\end{enumerate}

\subsection{Functions}
Recall that a function is a rule that assigns an output value to some input value.  For example, a rule might be to take the input (a real number) and output its square (also a real number).  We can call this rule $f$ and write:
\begin{equation*}
f(x) = x^2
\end{equation*}
Mathematicians like to be clear on the input (domain) of a function and the output (range) of a function.  Because our $f$ above can take any real number and will output a real number, we say that $f$ is a real-valued function of a real variable.  That is a lot to say.  We can use a shorthand, by denoting the set of real numbers by $\mathbb{R}$.  We then write:
\begin{equation}
f:\mathbb{R}\rightarrow \mathbb{R}
\end{equation}
This says the same thing as '$f$ is a real-valued function of a real variable', but it is shorter, and we may read it as '$f$ is a function from $\mathbb{R}$ to $\mathbb{R}$'.  You will encounter functions on other sets, for example, $\mathbb{N}$ - the natural numbers, $\mathbb{Z}$ - the integers, etc.  We will mostly focus on real functions of real variables for the purposes of these notes. 

\end{document}