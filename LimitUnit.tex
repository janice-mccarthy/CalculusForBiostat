
\documentclass[12pt,a4paper]{article} % Use A4 paper with a 12pt font size - different paper sizes will require manual recalculation of page margins and border positions

% Generated with LaTeXDraw 2.0.8
% Mon Jun 17 19:00:40 EDT 2013
\usepackage[usenames,dvipsnames]{pstricks}
\usepackage{epsfig}
\usepackage{pst-grad} % For gradients
\usepackage{pst-plot} % For axes

\usepackage[left=1.3cm,right=4.6cm,top=1.8cm,bottom=4.0cm,marginparwidth=3.4cm]{geometry} % Adjust page margins
\usepackage{amsmath} % Required for equation customization
\usepackage{amssymb} % Required to include mathematical symbols
\usepackage{xcolor} % Required to specify colors by name
\usepackage{amsthm}
\usepackage{float}


\setlength{\parindent}{0cm} % Remove paragraph indentation
\newcommand{\tab}{\hspace*{2em}} % Defines a new command for some horizontal space


\title{Calculus Review - Limits}
%----------------------------------------------------------------------------------------

\newtheorem{defn}{Definition}
\newtheorem{example}{Example}
\newtheorem{prop}{Proposition}
\newtheorem{exer}{Exercises}
\newtheorem{thm}{Therorem}
\begin{document}
\maketitle

\section{Limits} 
Many calculus texts begin with a version of "Zeno's Paradox", that goes something like this: If one were to shoot an arrow at a tree, at some point, the arrow is 1/2 the distance from the tree, and then 1/2 + 1/4, and then 1/2 + 1/4 + 1/8. We can keep halving the distance, but we'll never (in finite time) get to 0, and therefore the distance is never reached and the arrow doesn't hit the tree. But of course, the arrow \emph{does} reach the tree. We therefore have a paradox.\\\\

In mathematics, we often resolve paradoxes by defining something or setting some requirement. For example, when defining sets, Russel (and others in parallel) found that they couldn't just say a set is \emph{any} collection of objects, because that led to a paradox. The result was a set of 'axioms' that define all the properties a set must have.\\\\

For the case of Zeno's paradox, the notion of a limit was developed. We can write the sum we want to compute this way:

$$\sum\limits_{n=0}^\infty \frac1{2^n}$$

\vspace{0.2in}
If we want to actually compute this, or even show that it's a finite number, we need to define limits. Our intuition is that the sum should add to 1, and the amount we add should eventually go to zero. \\\\
When we talk about the limit $L$ of a function, say $f(x)$, as $x$ approaches some value $a$, we can think of it this way: as $x$ gets closer to $a$, $f(x)$ should get closer to $L$.

$$\lim\limits_{x\rightarrow a} f(x) = L$$

We need to turn this into a definition that allows us to prove certain properties of limits, and to use those to make new definitions (such as derivatives and integrals). First, let's have a look at the formal definition, and then we'll take it apart step-by-step to see what it means.

\begin{defn}
Let $f$ be a real-valued function of a real variable, $x$.  Let $a,L$ be fixed real numbers.  If for every $\epsilon >0$, there exists some $\delta>0$ such that
\begin{equation*}
|f(x) - L|<\epsilon 
\end{equation*}
whenever
\begin{equation*}
|x - a|<\delta 
\end{equation*}
we say that
\begin{equation*}
\lim_{x\rightarrow a} f(x) = L
\end{equation*}
\end{defn}
 One of the things that can be confusing about this definition is that it looks bit backwards: We want $x$ close to $a$ to mean that $f(x)$ is close to $L$, but we start the definition talking about $f(x)$ and $L$. We have to reframe our intuitive statement a bit so that it lines up with our $\delta$ and $\epsilon$ nonsense.\\
 
Instead of thinking of $x$ approaching $a$, we start by specifying \emph{how close we would like $f(x)$ to be to $L$}, and we ask how close $x$ needs to be to $a$ to make that possible. So we specify that we want $f(x)$ to be 'within $\epsilon$' of $L$, i.e. we want 

$$|f(x) - L| < \epsilon.$$
 
We are saying that given any specified tolerance ($\epsilon$), which can be arbitrarily small, we can find a distance $\delta$ so that if $x$ is within $\delta$ of $a$, then $f(x)$ will be within $\epsilon$ of $L$. 


Loosely speaking, if $x$ is close enough to $a$, $f(x)$ will be close to $L$.  Pictorally:

\includegraphics[height=3in]{limit.pdf}

The function pictured has a special property: it is \emph{continuous}.  We will give a formal definition of this in a moment, but it means that the limit of the function at a point is equal to the function value.  Let us look at another example, one that gives a little better idea of the utility of such a thing as a limit.  Consider\\

\begin{equation}
\lim_{x\rightarrow 0}\frac{\sin(x)}{x}
\end{equation}
This function is not defined at $x=0$, however, the above limit exists!  We can see this by looking at the graph of $\frac{\sin(x)}{x}$.  

\includegraphics[height=3in]{sinxoverx.pdf}

The graph does not constitute a proof, but we can see that the graph approaches $1$ as $x\rightarrow 0$ from both the left and the right.    How \emph{do} we prove limits?


This brings us to our next topic:

\subsection{Proving Limits with $\epsilon$ and $\delta$}

In the previous section, we stated the formal definition of a limit.  When we wish to prove a limit, we may start with this definition.
\begin{example}
Prove that $$\lim_{x\rightarrow 3} 2x +4 =10$$.\\
\end{example}
The formal definition of a limit says that given any $\epsilon>0$, we must find $\delta>0$ such that $|2x+4 -10|<\epsilon$ whenever $|x-3|<\delta$.  When we prove a limit from the definition, we must always find an expression for $\delta$ in terms of $\epsilon$.  We begin with the inequality containing $\epsilon$ and 'work backwards':
\begin{eqnarray*}
|2x+4-10| &<& \epsilon \iff \\
|2x-6|&<& \epsilon \iff \\
2|(x-3)|&<&\epsilon \iff\\
|(x-3)|&<&\frac{\epsilon}2
\end{eqnarray*}
So we may set $\delta = \frac{\epsilon}{2}$.  Now we are ready for the proof:
\begin{prop}
$$\lim_{x\rightarrow 0} 2x + 4 =10$$
\end{prop}
\begin{proof}
Fix $\epsilon>0$ and set $\delta = \frac{\epsilon}{2}$.  Then if $|x-3|<\delta$:
\begin{eqnarray*}
|x-3|&<&\frac{\epsilon}{2}\iff\\
2|x-3|&<&\epsilon\iff\\
|2x -6|&<&\epsilon\iff\\
|2x +4 -10|&<&\epsilon
\end{eqnarray*}
\end{proof}

The 'working backwards' part is usually what is most confusing in the beginning.  It takes practice!  Here are some exercises:
\begin{exer}
Prove the following limits
\begin{enumerate}
\item $$\lim_{x\rightarrow 4} 7x+3=31$$
\item $$\lim_{x\rightarrow -3} x+5 =2$$
\item $$\lim_{x\rightarrow 1}{\frac{4}{5}x +\frac15=1}$$
\end{enumerate}
\end{exer}
Those examples were all \emph{linear} functions of $x$, and therefore have a certain simplicity.  Let's try another example:
\begin{example}
Prove $$\lim_{x\rightarrow 3} x^2 =9$$.
\end{example}
As before, we start with the definition and note that given $\epsilon>0$, we must find some $\delta>0$ so that $|x^2-9|<\epsilon$ whenever $|x-3|<\delta$.  We start with the $\epsilon$ term and work backwards:
\begin{eqnarray*}
|x^2-9|&<&\epsilon\iff\\
|(x+3)(x-3)|&<&\epsilon\iff\\
|(x+3)||(x-3)|&<&\epsilon\iff\\
\end{eqnarray*}
Now what?  In the previous examples, we could just 'solve' the inequality for the $|x-a|$ expression and get $\delta$ in terms of $\epsilon$.  We cannot do that here, because we would need $\delta=\frac{\epsilon}{|x+3|}$ and we cannot have $\delta$ in terms of $x$.  What we need to do here is to find a way to bound $|x+3|$.  The trick is that we are specifying $x$ 'close' to $3$ in choosing $\delta$.  We choose an arbitrary $\delta$, say $\delta=1$.  Now, if $|x-3|<1$:
\begin{equation*}
-1<x-3 <1 \iff
\end{equation*}
\begin{equation*}
 2<x   <4
\end{equation*}    
This means we can bound the $|x+3|$ term, when $|x-3|$ is less than $1$:
\begin{eqnarray*}
|x-3| &<& 1\Rightarrow\\
|x+3| &<& 7 
\end{eqnarray*}
Now, we are ready for the proof. We will set $\delta =\mathrm{min}\left(1,\frac{\epsilon}{7}\right)$, so that we know we have set $|x-3|$ to be \emph{at most} smaller than $1$, and our bound for $|x+3|$ above is valid:
\begin{prop}
$$\lim_{x\rightarrow 3}x^2 = 9$$
\end{prop} 
\begin{proof}
Let $\epsilon>0$ be fixed and set $\delta =\mathrm{min}\left(1,\frac{\epsilon}{7}\right)$.  (Note that this ensures that $|x-3|$ is both $<1$ \emph{and} $<\frac{\epsilon}{7}$ Then:
\begin{eqnarray*}
|x-3|&<&\delta\Rightarrow\\
|x-3|&<1& \Rightarrow\\
|x+3| &<& 7 
\end{eqnarray*}
So, we have that:
\begin{eqnarray*}
|x^2-9|&=&|x+3||x-3|\\
&<& 7|x-3|\\
&<& \epsilon
\end{eqnarray*}
\end{proof}
And now, it's your turn:
\begin{exer}
Prove the following limits:
\begin{enumerate}
\item $$\lim_{x\rightarrow 2} x^2-4=0$$
\item $$\lim_{x\rightarrow 2} x^2-3x+1=-1$$
\end{enumerate}
\end{exer}
\subsection{Limits at Infinity}
We now move on to defining limits at infinity.  We would like to understand the behavior of a function, say $f(x)$ as $x$ gets arbitrarily large.  Will $f(x)$ also increase indefinitely?  Will it decay to some constant value?  Or perhaps $f(x)$ will not achieve a limit at all.  How can we tell?  First we need a formal definition, so that we know \emph{exactly} what we mean when we talk about the limit of a function as $x\rightarrow\infty$:
\begin{defn}
Let $f:\mathbb{R}\rightarrow\mathbb{R}$. Let $L$ be a fixed real number.  If for every $\epsilon >0$, there exists some number $c>0$ such that
\begin{equation*}
|f(x) - L|<\epsilon 
\end{equation*}
whenever
\begin{equation*}
x>c 
\end{equation*}
we say that
\begin{equation*}
\lim_{x\rightarrow \infty} f(x) = L
\end{equation*}
\end{defn}
Now let's use this definition to prove a limit:
\begin{example}
Prove $$\lim_{x\rightarrow\infty} \frac{1}{x} = 0$$
\end{example}
Again, we work backwards.  Given some fixed $\epsilon>0$, we want to find some value $c$ so that if $x>c$, then $|\frac{1}{x}|<\epsilon$.  Without loss of generality, we can assume $x>0$ (we want to work with large values of $x$).  
\begin{eqnarray*}
\frac{1}{x}&<\epsilon \Rightarrow\\
x&>\frac{1}{\epsilon}
\end{eqnarray*}
so we simply set $c=\frac{1}{\epsilon}$.  Now we write the proof:
\begin{prop}
$$\lim_{x\rightarrow\infty} \frac{1}{x} = 0$$
\end{prop}
\begin{proof}
Let $\epsilon>0$ be fixed and set $c=max(0,\frac{1}{\epsilon})$.  Then:
\begin{eqnarray*}
x&>c\Rightarrow\\
x&>\frac{1}{\epsilon}\Rightarrow
\frac{1}{x}&<\epsilon
\end{eqnarray*}
\end{proof}
\begin{exer}
Prove the following limits:
\begin{enumerate}
\item $$\lim_{x\rightarrow\infty} \frac{8x+2}{4x+1} = 2$$
\item $$\lim_{x\rightarrow\infty} \frac{\cos(x)}{x} = 0$$
\item $$\lim_{x\rightarrow\infty} \frac{2x}{3x-1} = \frac23$$
\end{enumerate}
\end{exer}
\end{document}